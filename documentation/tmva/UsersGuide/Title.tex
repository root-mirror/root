\vspace{-1cm}
\begin{flushright}
{\sf\em arXiv:physics/0703039 [Data Analysis, Statistics and Probability]} \\
{\sf\em CERN-OPEN-2007-007} \\
{\sf\em TMVA version \TMVAVersion} \\
{\sf\em \today} \\
\def\UrlFont{\sf\em}
\url{http://tmva.sourceforge.net} 
\end{flushright}
\def\UrlFont{\rm}

\def\miniPageOffset{0.2cm}
\def\miniPageWidth{13.5cm}
\vspace*{\stretch{20}}
\HRule
\begin{flushleft}
\hspace{\miniPageOffset}\begin{minipage}{\miniPageWidth}
{\sf\Huge\bfseries\boldmath TMVA 4} \\[0.2cm]
{\sf\Large\bfseries\boldmath Toolkit for Multivariate Data Analysis with ROOT} \\[1cm]
{\sf\Huge\bfseries\boldmath Users Guide} 
\end{minipage}
\end{flushleft}
\HRule
\vspace{2.0cm}
%\begin{flushleft}
\begin{flushright}
% \hspace{\miniPageOffset}\begin{minipage}{\miniPageWidth}
{\sf\Large  A.~Hoecker,~P.~Speckmayer,~J.~Stelzer,~J.~Therhaag,~E.~von Toerne,~H.~Voss} 
% \end{minipage}

\vspace{1.2cm}
% \hspace{\miniPageOffset}\begin{minipage}{\miniPageWidth}
{\sf\em\large Contributed to TMVA have:} \\[0.4cm]
{\sf\large 
M.~Backes,
T.~Carli,
O.~Cohen,
A.~Christov, 
D.~Dannheim,
K.~Danielowski,\\[0.1cm]
S.~Henrot-Versill\'e, 
M.~Jachowski, 
K.~Kraszewski,
A.~Krasznahorkay Jr.,  \\[0.1cm]    
M.~Kruk,
Y.~Mahalalel, 
R.~Ospanov, 
X.~Prudent, 
A.~Robert,
C.~Rosemann, \\[0.1cm]  
D.~Schouten, 
F.~Tegenfeldt,
A.~Voigt,
K.~Voss,
M.~Wolter, 
A.~Zemla,
J.~Zhong, \\[0.1cm]
A.~Moudgil,
K.~Albertsson
}

% \end{minipage}
\vspace*{\stretch{1}}
\end{flushright}
%\end{flushleft}

\vfill

\thispagestyle{empty}
\newpage

% authors


\def\UrlFont{\sf}
{\small\sf
{\sf\bfseries Abstract} --- 
In high-energy physics, with the search for ever smaller signals in
ever larger data sets, it has become essential to extract a maximum of
the available information from the data.  Multivariate classification
methods based on machine learning techniques have become a fundamental
ingredient to most analyses.  Also the multivariate classifiers
themselves have significantly evolved in recent years. Statisticians
have found new ways to tune and to combine classifiers to further gain
in performance. Integrated into the analysis framework ROOT, TMVA is
a toolkit which hosts a large variety of multivariate classification
algorithms. Training, testing, performance evaluation and application of 
all available classifiers is carried out simultaneously via user-friendly 
interfaces. With version 4, TMVA has been extended to multivariate 
regression of a real-valued target vector. Regression is invoked through 
the same user interfaces as classification. TMVA 4 also features 
more flexible data handling allowing one to arbitrarily form 
combined MVA methods. A generalised boosting method is the first 
realisation benefiting from the new framework.
}
\vspace{0.5cm}
\begin{center}
{\small\sf
{\sf\bfseries TMVA \TMVAVersion\ -- Toolkit for Multivariate Data Analysis with ROOT}  \\
Copyright\index{Copyright} 
\copyright\  2005-2018, Regents of 
CERN (Switzerland),  
DESY (Germany),
MPI-Kernphysik Heidelberg (Germany),
University of Bonn (Germany),
and University of Victoria (Canada). \\
BSD license: \url{http://tmva.sourceforge.net/LICENSE}. 

{\sf\bfseries Authors:} \\
Andreas Hoecker (CERN, Switzerland) \emailsf{<andreas.hoecker@cern.ch>}, \\
Peter Speckmayer (CERN, Switzerland) \emailsf{<peter.speckmayer@cern.ch>}, \\
J\"org Stelzer (CERN, Switzerland) \emailsf{<joerg.stelzer@cern.ch>},\\
Jan Therhaag (Universit\"at Bonn, Germany) \emailsf{<therhaag@physik.uni-bonn.de>}, \\
Eckhard von Toerne (Universit\"at Bonn, Germany) \emailsf{<evt@physik.uni-bonn.de>},\\
Helge Voss (MPI f\"ur Kernphysik Heidelberg, Germany) \emailsf{<helge.voss@cern.ch>},\\
\hspace{0.5cm} \\
Moritz Backes, % (Geneva University, Switzerland) \emailsf{moritz.backes@cern.ch},\\
Tancredi Carli, % (CERN, Switzerland) \emailsf{<tancredi.carli@cern.ch>}, \\
Or Cohen, % (CERN, Switzerland and Technion, Israel) \emailsf{<or.cohen@cern.ch>}, \\
Asen Christov, % (Universit\"at Freiburg, Germany) \emailsf{<christov@physik.uni-freiburg.de>}, \\
Krzysztof Danielowski, % (IFJ and AGH/UJ, Krakow, Poland) \emailsf{<Krzysztof.Danielowski@cern.ch>}, \\
Dominik Dannheim, % (CERN, Switzerland) \emailsf{<Dominik.Dannheim@cern.ch>}, \\
Sophie Henrot-Versill\'e, % (LAL Orsay, France) \emailsf{<versille@lal.in2p3.fr>}, \\
Matthew Jachowski, % (Stanford University, USA) \emailsf{<jachowski@stanford.edu>}, \\
Kamil Kraszewski, % (IFJ and AGH/UJ, Krakow, Poland) \emailsf{<kamil.bartlomiej.kraszewski@cern.ch>}, \\
Attila Krasznahorkay Jr., % (CERN, CH, and Manchester U., UK)  \emailsf{<Attila.Krasznahorkay@cern.ch>}, \\
Maciej Kruk, % (IFJ and AGH/UJ, Krakow, Poland) \emailsf{<maciej.mateusz.kruk@cern.ch>}, \\
Yair Mahalale, %l (Tel Aviv University, Israel) \emailsf{<yair@mahalalel.com>}, \\
Rustem Ospanov, % (University of Texas, USA) \emailsf{<rustem@fnal.gov>}, \\
Xavier Prudent, % (LAPP Annecy, France) \emailsf{<prudent@lapp.in2p3.fr>}, \\
Doug Schouten, % (S. Fraser U., Canada) \emailsf{<dschoute@sfu.ca>}, \\
Fredrik Tegenfeldt, % (Iowa University, USA) \emailsf{<fredrik.tegenfeldt@cern.ch>}, \\
Arnaud Robert, % (LPNHE Paris, France) \emailsf{<arobert@lpnhe.in2p3.fr>}, \\
Christoph Rosemann, % (DESY, Germany) \emailsf{<christoph.rosemann@desy.de>}, \\
Alexander Voigt, % (CERN, Switzerland) \emailsf{<alexander.voigt@cern.ch>}, \\
Kai Voss, % (University of Victoria, Canada) \emailsf{<kai.voss@cern.ch>}, \\
Marcin Wolter, % (IFJ PAN Krakow, Poland) \emailsf{<marcin.wolter@ifj.edu.pl>}, \\
Andrzej Zemla, % (IFJ PAN Krakow, Poland) \emailsf{<zemla@corcoran.if.uj.edu.pl>}, \\
Jiahang Zhong, % (Academia Sinica, Taipei, Taiwan) \emailsf{<Jiahang.Zhong@cern.ch>}, \\
Abhinav Moudgil, % (IIIT Hyderabad, India) \emailsf{<abhinav.moudgil@research.iiit.ac.in>}, \\
Kim Albertsson % (LTU/CERN, Sweden/Geneva) \emailsf{<kim.albertsson@cern.ch>}, \\
\\
\hspace{0.5cm} \\
and valuable contributions from many users, please see acknowledgements.
}
\end{center}

\thispagestyle{empty}
\newpage

%%% Local Variables: 
%%% mode: latex
%%% TeX-master: "TMVAUsersGuide"
%%% End: 
